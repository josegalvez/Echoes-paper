\pdfoutput=1
\documentclass[article,aps,nofootinbib,twocolumn,superscriptaddress]{revtex4-1}
\input epsf
\usepackage{graphics}
\usepackage{amsmath}
\usepackage{amssymb}
\usepackage{bm}
\usepackage{braket}
\usepackage{booktabs}
\usepackage{subfigure}
\usepackage{subfloat}
\usepackage{float}
\usepackage[usenames,svgnames]{xcolor}
\usepackage{hyperref}

\hypersetup{
    colorlinks=true,
    urlcolor=SteelBlue,
    linkcolor=red,
    citecolor=blue,
}


\usepackage{color}
\usepackage{dcolumn}
\usepackage{hyphenat}


\def\be{\begin{equation}}
\def\ee{\end{equation}}
\def\ba{\begin{eqnarray}}
\def\ea{\end{eqnarray}}

\newcommand{\ack}[1]{[{\bf Pfft!#1}]}
\newcommand{\tvb}[1]{{\bf \color{blue}{{#1}}}}
\newcommand{\tvg}[1]{{\bf \color{green}{{#1}}}}
\newcommand{\tvr}[1]{{\bf \color{red}{{#1}}}}
\newcommand{\ay}[1]{{\bf \color{magenta}{{#1}}}}

%\maketitle
\usepackage{graphicx}

\begin{document}

\title{Producing echoes from the scattering of wavepackets on wormholes}

\author{Jos\'e T. G\'alvez Ghersi}
\email{joseg@sfu.ca}
\author{Andrei V. Frolov}
\email{frolov@sfu.ca}
\author{David Dobre}
\email{ddobre@sfu.ca}
\affiliation{Department of Physics, Simon Fraser University, Burnaby, BC, V5A 1S6, Canada}


\begin{abstract}
In the light of the recent progress showing that the hypothetical observation of pulses isolated from the gravitational radiation transient (also known as echoes) would prove the existence of exotic compact objects (ECOs); it is possible to reproduce many features of the ringdown signal by simulating a scattering problem instead of the full coalescence of ECOs. In this paper, we study the dynamics of scalar and tensor wavepackets colliding against a spherically symmetric traversable wormhole. Our purpose is to extract the features of the time dependent scattering solutions inside and outside the effective potential cavity in addition to their asymptotic behaviour. Using the geometrical optics approximation, we show that the amplitude of the echoes is maximum only in a narrow bandwidth of frequency space, where the intensity of the transient is reduced. The computer code used to produce these results is publicly available for further applications, including scattering and accretion processes.      
\end{abstract}

\maketitle


\section{Introduction}
The era of gravitational wave (GW) astronomy \citep{Abbott:2016blz, Abbott:2016nmj} has begun. GW spectroscopy, in contrast to its atomic counterpart, allows us to explore a completely different radiative regime, in which it is possible to explore the role of strongly interacting gravitational degrees of freedom in a wide range of astrophysical and cosmological phenomena. Furthermore, the long absence of observational evidence confirming the dynamical properties of spacetime has motivated plausible conjectures about the behaviour of gravity within and beyond the boundaries of General Relativity (GR), these hypotheses can be tested once the detectors have reached adequate sensitivity levels. 
\bibliography{bibliography.bib}

\end{document}
