\pdfoutput=1
\documentclass[article,aps,nofootinbib,twocolumn,superscriptaddress]{revtex4-1}
\input epsf
\usepackage{graphics}
\usepackage{amsmath}
\usepackage{amssymb}
\usepackage{bm}
\usepackage{braket}
\usepackage{booktabs}
\usepackage{subfigure}
\usepackage{subfloat}
\usepackage{float}
\usepackage[usenames,svgnames]{xcolor}
\usepackage{hyperref}

\hypersetup{
    colorlinks=true,
    urlcolor=SteelBlue,
    linkcolor=red,
    citecolor=blue,
}


\usepackage{color}
\usepackage{dcolumn}
\usepackage{hyphenat}


\def\be{\begin{equation}}
\def\ee{\end{equation}}
\def\ba{\begin{eqnarray}}
\def\ea{\end{eqnarray}}

\newcommand{\ack}[1]{[{\bf Pfft!#1}]}
\newcommand{\tvb}[1]{{\bf \color{blue}{{#1}}}}
\newcommand{\tvg}[1]{{\bf \color{green}{{#1}}}}
\newcommand{\tvr}[1]{{\bf \color{red}{{#1}}}}
\newcommand{\ay}[1]{{\bf \color{magenta}{{#1}}}}

%\maketitle
\usepackage{graphicx}

\begin{document}

\title{Echoes from the scattering of wavepackets on wormholes}

\author{Jos\'e T. G\'alvez Ghersi}
\email{joseg@sfu.ca}
\author{Andrei V. Frolov}
\email{frolov@sfu.ca}
\author{David Dobre}
\email{ddobre@sfu.ca}
\affiliation{Department of Physics, Simon Fraser University, Burnaby, BC, V5A 1S6, Canada}


\begin{abstract}
In the light of the recent progress showing that the hypothetical observation of pulses isolated from the gravitational radiation transient (also known as echoes) would prove the existence of exotic compact objects (ECOs); it is possible to reproduce many features of the ringdown signal by simulating a scattering problem instead of the full coalescence of ECOs. In this paper, we study the dynamics of scalar and tensor wavepackets colliding against a spherically symmetric traversable wormhole. Our purpose is to extract the features of the time dependent scattering solutions inside and outside the effective potential cavity in addition to their asymptotic behaviour. Using the geometrical optics approximation, we show that the amplitude of the echoes is only considerable in a narrow bandwidth of frequency space, where the intensity of the transient is reduced. The computer code used to produce these results is publicly available for further applications, including scattering and accretion processes.      
\end{abstract}

\maketitle


\section{Introduction}
The era of gravitational wave (GW) astronomy \citep{Abbott:2016blz, Abbott:2016nmj} has begun. GW spectroscopy, in contrast to its atomic counterpart, allows us to charcterize strong gravitational interactions in their radiative regime. In this new range of frequencies, it is now possible to explore the role of dynamical gravitational degrees of freedom in a wide range of astrophysical \citep{Frolov:2017asg, Cardoso:2016rao} and cosmological \citep{Krauss989, Ade:2018gkx} phenomena. 

The long absence of observational evidence confirming the dynamical properties of spacetime has motivated a plethora of conjectures about the behaviour of gravity within and beyond \citep{Clifton:2011jh, Taliotis:2012sx, Joyce:2014kja} classical General Relativity (GR). Latterly, the potential existence of exotic compact objects (ECOs) sourced by quantum effects on gravity \citep{PhysRevLett.61.1446, Almheiri:2012rt, Mazur:2001fv} (such as wormholes, firewalls and gravastars) has captured the attention of many recent efforts \citep{Cardoso:2016oxy, Abedi:2016hgu, Abedi:2018npz}, wherein the principal claim is that the detection of a train of ``echoes'' isolated from the main transient of GW and with generically large amplitudes would be a clear evidence of ECOs. It is, therefore, necessary to understand (i) the mechanisms behind the production of echoes and (ii) the intensity and spectrum of the outgoing wavelets compared to the GW transient in the simplest possible setup. In this paper, we explore the generation of echoes by colliding wavepackets of scalar and tensor radiation against a traversable spherically symmetric wormhole \citep{Visser:1989kh}; which behaves just like a Fabry-Perot cavity, shares common properties with the effective potential cavities made by other ECOs, like gravastars and firewalls, and the main features of the dispersed pulses are similar to the ringdown signals after the coalescence of ECOs.

Here we consider a wormhole configuration made by the junction of two Schwarzschild geometries of identical mass at $r_0>2M$, just as shown in \citep{Cardoso:2016oxy}. In this case, the symmetry of the centrifugal barriers at $r=3M$ on each side of the throat allows us to find the reflection and transmission coefficients of the cavity. Hence, it is possible to reconstruct the spectral shape of the outgoing pulse using the geometrical optics approximation. Nevertheless, this approximation predicts an exponential decay of the subsequent higher order reflections, which appears instead as a power law in the full solution of the scattering problem. Thus, the time evolving profile of the echoes can only be generated by the excitation of quasinormal modes (QNMs), these modes are sourced by a sequence of internal reflections inside the potential cavity and then propagate throughout the surface of the maximal potential energy spheres (i.e. the ``edges'' of the potential barriers), while leaking energy into the exterior. Quasinormal modes of the Schwarzschild solution have been extensively studied and reproduced in various analytic and numerical simulations \citep{Chandrasekhar:1975zza, PhysRevD.46.4179}, thus it is easy to identify their characteristic frequencies in the spectrum of outgoing pulses. We as well present the full scattering solution both inside and outside the wormhole cavity in detail, along with the energy fluxes and the asymptotic solutions for the principal spherical modes of a scalar (and tensor) wavepackets. In addition to this, we find the width and frequency intervals contained in the incident wavepackets for which the outgoing the wavelets have maximal amplitudes.

The layout for this paper is as follows: 
\bibliography{bibliography.bib}

\end{document}
