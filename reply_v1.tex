\documentclass[preprintnumbers,floats,floatfix,amssymb,prd,onecolumn,superscriptaddress,nofootinbib]{revtex4}
%\documentclass[preprintnumbers,showpacs,floats,twocolumn,prd,aps]{revtex4}
\usepackage{amssymb,amsmath}
\usepackage{epsfig,hyperref}
\usepackage[svgnames]{xcolor}
\usepackage{pgf,tikz}
\usepackage{epstopdf}
\usepackage[british]{babel}

\makeatletter
\DeclareRobustCommand*{\bfseries}{%
  \not@math@alphabet\bfseries\mathbf
  \fontseries\bfdefault\selectfont
  \boldmath
}
\makeatother

%%
\special{papersize=8.5in,11in}

\def\be{\begin{equation}}
\def\ee{\end{equation}}
\def\beq{\begin{eqnarray}}
\def\eeq{\end{eqnarray}}

\makeatletter
\renewcommand{\vec}[1]{\mbox{\boldmath$#1$}}
\newcommand{\arXiv}[2][]{\href{http://arxiv.org/abs/#2}{\texttt{arXiv:#2\@ifempty{#1}{}{ [#1]}}}}
\makeatother

\begin{document}
\title{Reply to referees for CQG-105854}

\author{Jose T. Galvez Ghersi}%
\email{joseg@sfu.ca}
\affiliation{Department of Physics, Simon Fraser University, Burnaby, British Columbia, V5A 1S6, Canada}

\author{Andrei V. Frolov}
\email{frolov@sfu.ca}
\affiliation{Department of Physics, Simon Fraser University, Burnaby, British Columbia, V5A 1S6, Canada}

\author{David A. Dobre}
\email{ddobre@sfu.ca}
\affiliation{Department of Physics, Simon Fraser University, Burnaby, British Columbia, V5A 1S6, Canada}
\date{\today}

\begin{abstract} 
Summary: Thanks to the valuable comments of the referees, we had the chance to review our calculations and solve any potential source of confusion. We have updated our draft to accommodate all the comments and objections made to the previous version of our manuscript. We worked throughout the following points: (1) Emphasis on the generation of a small net polarization as one of the main results of our paper, by mentioning this finding from the beginning of the manuscript. (2) We included all the suggested references in both the text and the bibliography. (3) The geometrical optics reconstruction formula was generalized to consider walls with different reflectivities and transmissivities. (4) The concept of reflection and transmission coefficients only refer to an individual potential barrier. (5) We also mentioned the ``pinching off'' of the Morris-Thorne solution in larger timescales. (6) We simulated the scattering of a Gaussian pulse of GW towards a ``longer'' wormhole throat (the length of the throat is now $100 r_g$), finding again that the magnitude of the outgoing pulse is not generically large. 
\end{abstract}
\maketitle

\section{Referee 1}

\textbf{Comment:} ...I struggled to find what is the main original contribution of this manuscript... In the end, I concluded that the remarks on possible polarization changes are potentially the most significant outcome of their efforts.\\

\textbf{Reply:} We agree that the small net polarization of GW is one of the main results of our manuscript, and we should include it in the body of the paper instead of being in an Appendix. Due to this, this is now appearing in Section III. Additionally, we included it in the introduction of our paper as one of our results.\\
  
\textbf{Comment:} ... the authors must address the issues with citation of the extensive literature on the subject and place their own efforts in the context of those works. In particular, the authors should note the following papers and comment on how their own results fit with these.\\  
\begin{itemize}
\item{https://arxiv.org/abs/1706.06155}\\
\item{https://arxiv.org/abs/1702.04833}\\
\item{https://arxiv.org/abs/1711.00391}\\
\item{https://arxiv.org/abs/1802.02003}\\
\item{https://arxiv.org/abs/1802.07735}\\
\item{https://arxiv.org/abs/1806.04253}\\
\item{https://arxiv.org/abs/1810.07137}\\
\item{https://arxiv.org/abs/1902.08180}\\
\end{itemize}


\textbf{Reply:}
Thanks for bringing these papers to our attention. All the references suggested were added in the latest draft of our paper (in addition to some others). It is interesting to notice that the Morris-Thorne wormhole eventually evolves into a black hole, as noticed by Wang et al. in arxiv:1802.02003. In addition to this, the idea of a Dyson-series expansion presented by Correia et al. in arXiv:1802.07735 is related to the geometric expansion we suggested in our manuscript.\\ 

\section{Referee 2}

\textbf{Comment:} In Eqs. (12) and (13), the authors seems to assume that the reflection coefficient $R$ of the potential barrier for the wave incident from right to left is equivalent to that for the wave incident from left to right. Generally, both are unequal. The authors should refine Eqs. (12) and (13).\\

\textbf{Reply:} We derived in Eq. (12) a more general expression for the geometrical series, now considering the different reflection and transmission coefficients for each potential wall. Furthermore, we used these general expressions to derive the reconstruction series for a system with two identical walls in Eq. (14).\\
 
\textbf{Comment:} $R$ and $T$ in Eq. (21) are valid for single barriers, but for a couple of barriers, I am not convinced that (21) is valid..\\  
  
\textbf{Reply:} We agree with the referee on this comment, we have now clarified that Eq.(21) only refers to the coefficients of each individual wall, but not the whole system. Nevertheless, this definition is sufficient to perform the analysis followed in our manuscript since the system we study is constituted by two identical walls.\\

\textbf{Comment:} It is interesting to consider the case that the MT wormhole is slowly pinching off and eventually collapses into a black hole (so the increasing of echo intervals with the time, see arXiv:1802.02003). The authors should have a mention.   \\

\textbf{Reply:} Thanks, this result is now mentioned (and referenced) in our draft.\\

\textbf{Comment:} In the tortoise coordinate, the distance between the barriers of MT wormhole depends on the radius of throat, see Eq.(4) in e.g.arXiv:1802.02003, (if it equals to the 
Schwarzschild radius, this distance is infinite). However, in your Figs. 5 and 7, the distance you choose is $<100$. If this number is larger, does your result (the amplitude of the echoes is only large enough in a narrow bandwidth of frequency space) alter?\\

\textbf{Reply:}  We have simulated the collision of a wavepacket (and the corresponding time evolution plot) towards a cavity with a longer throat distance of 100 $r_g$ (more than two times bigger than the original length), we chose an arbitrary initial Gaussian pulse with $\sigma=16r_g$. Showing in Fig. 17 -- before Subsection IIA -- that the amplitude of the outgoing echoes is three orders of magnitude smaller than the transient, which is coherent with our result. All of these details have been incorporated in the text of our latest draft.\\



\end{document}
